%%This is a very basic article template.
%%There is just one section and two subsections.
\documentclass[11pt,a4paper,oneside]{report}

\usepackage{listings}
\usepackage{fullpage}
\usepackage{amsmath}

\usepackage{listings}
\usepackage{color}
\usepackage{textcomp}
\definecolor{listinggray}{gray}{0.9}
\definecolor{lbcolor}{rgb}{0.9,0.9,0.9}
\lstset{
  backgroundcolor=\color{lbcolor},
  tabsize=4,
  rulecolor=,
  language=C,
  basicstyle=\scriptsize,
  upquote=true,
  aboveskip={1.5\baselineskip},
  columns=fixed,
  showstringspaces=false,
  extendedchars=true,
  breaklines=true,
  prebreak = \raisebox{0ex}[0ex][0ex]{\ensuremath{\hookleftarrow}},
  frame=single,
  showtabs=false,
  showspaces=false,
  showstringspaces=false,
  identifierstyle=\ttfamily,
  keywordstyle=\color[rgb]{0,0,1},
  commentstyle=\color[rgb]{0.133,0.545,0.133},
  stringstyle=\color[rgb]{0.627,0.126,0.941},
}

\begin{document}
\lstset{language=C}
\setcounter{secnumdepth}{1}


\title{\center{Data Networks project}}
\author{Group 7: \\ Igor Stassiy, Kirill Afanasev,Sanjar Karaev}
\date{\today}
\maketitle

\section*{Introduction}

  The problem - design a protocol which provides reliable service for
  in-order delivery of messages between hosts in multi-host networks with
  different topologies.  \\
  
  Assumptions:
  \begin{enumerate}
    \item In the network hosts and links do not fail
    \item The links have tendency to loose and corrupt frames
    \item Frames can arrive out of order
    \item The links have varying capacities and MTU
    \item Nodes are only aware of their immediate neighbors at the beginning
    \item Total number of hosts is at most 256 and their addresses are in range
    0-255
    \item The minimal possible MTU is 96 bytes
    \item The maximal possible message size generated by the application layer 
    is at most 10240 bytes 
  \end{enumerate}
  
\section*{Protocol Layers}
Our implementation consists of three layers - transport, network
and datalink layers. All of the layers are unaware of the message format of
the other and exchange the information only via an API described below. We
introduce a fixed API as well as a way to signal between layers, for example
telling transport layer that a congestion occured. This design allows to
easily extend functionality of layers.\\

\section*{Transport layer} 
The purpose of this layer is ensuring that the messages generated by the
application layer are delivered to the host in the correct order and in full.
It also uses flow control so that the receiver is not overwhelmed using a
sliding window for controlling messages to be sent. Both sender and receiver 
maintain a buffer of specified size in order to maintain the order of messages 
passed to the application layer.

Using ideas from IPv6, our protocol ensures that the messages are fragmented
only at the endpoints. Fragmentation and reassembly is managed entirely in the
transport layer. On message loss, only not acknowledged fragments are
retransmitted to save the bandwidth. Acknowledgements are used for signalling
that the whole range of fragments is received, which also saves the bandwidth
and avoids acknowledging separate fragments.
    
Services provided by the layer:
\begin{enumerate}
  \item Reliably delivers a packet from one host to another, avoiding
   packet loss, reordering and corruption.
   \item Accepting messages from application layer and passing down to the
   network layer
\end{enumerate}

\subsection{Flow control}
\noindent \textbf{Packet management}: Our transport layer manages flow control
with the help of sliding windows. There is a separate sliding window for each
``connection'' which completely separates packet flows between hosts. Both
receiver and sender maintain a fixed size sliding window, which allows higher
throughput and avoids dropping packets that can be used in future. Both receiver
and sender agree on the size of sliding window at the
beginning of transmission. \\
\noindent \textbf{Fragmentation}: We used ideas from IPv6, which avoids
fragmentation inside the network and delegates this responsibility to the
endpoints. Using network layer's service for path MTU discovery, sliding window
fragments packet in place and then treats each fragment as a separate packet. \\
\noindent \textbf{Fast retransmit, ACKs and NACKs}: Upon receiving a packet, the
receiver sends an acknowledgement. ACK has a sequence and segment numbers and an
ACK acknowledges a range of packets with sequence and segment number up to the
ones in the ACK packet. If there is fragmentation, ACK can be sent to
acknowledge a whole range of fragments which saves bandwidth. In order to save bandwidth
even more, the receiver piggybacks the acknowledgment with the packet it has to
send. If there is no such packet, it sends a separate ACK message. \\ When a
packet is not acknowledged for a specified time, a timer expires at the sender
and unacknowledged fragments are sent again. When an out of order packet (or
fragment) arrives this could be a sign of packet loss, so receiver sends back a
NACK to inform the sender quickly and sender uses fast retransmit of
possibly lost fragments. \\ 
Another strategy for detecting a lost packet is counting the number of ACKs
received for each packet (or fragment). In case the number of acks received for
a packet is more than 2, possibly a packet is lost and sender uses fast
retransmit of the packet. \\
\noindent \textbf{Timer management}
Absolutely essential for the transport layer to work quickly is the packet
retransmit timer. We use the Jacobson's algorithm for estimating round trip time 
of a packet and determining when to time out. $RTT$ is the estimated round trip
time, $SampleRTT$ is the current observed $RTT$ and $D$ is the standard
deviation of $RTT$ estimates:
\begin{align}
RTT & = \alpha \times RTT + (1-\alpha) \times SampleRTT \\
D & = \alpha \times D + (1-\alpha) \times \|RTT-SampleRTT\| \\
Timeout & = 2 \times RTT+4 \times D
\end{align}

$RTT$ is only sampled on packets that were not retransmitted. 

\noindent \textbf{Congestion control}
We use the Karn's algorithm to manage congestion control. More specifically
retransmit timer is multiplied by a constant larger than one each time a packet
timeout occurs. The technique is very effective as a retransmit is either
congestion in the network or packet loss. \\
\noindent \textbf{Compression}
Right after receiving message from application layer, we compress the message
and reassemble it in the transport layer at the endpoint. We use the $LZO$
compression algorithm for this purpose, a variant of the famous $LZ$ algorithm.
\noindent \textbf{Handling ambiguity}
The protocol suffers from the safe vulnerability as does the TCP layer. 
The following scenario describes when an error occurs:
\begin{enumerate}
\item a packet with sequence number $N$ is sent
\item the packet reaches the receiver
\item an ACK is sent by receiver
\item the ACK gets "astray'' in the network, by following a non-optimal route, and the packet is retransmitted by the sender
\item an ACK finally arrives at the sender
\item the sender's sliding window makes a full cycle
\item the retransmitted packet $N$ arrives at the receiver as its sequence number falls into the receivers window
\end{enumerate}

Clearly this ambuguity can only be treated with globally unique sequence numbers of messages which is not realistic.
A similar situation can arrise when the first ACK follows a non-optimal route. We treat the error by allowing
a large ration of allowed sequence numbers range and the sliding window size.

    \newpage
    Top level API:  
  \begin{lstlisting}
  // initialize transport layer 
  void init_transport();
  
  // write incoming message from network layer to transport layer
  void read_transport(PACKETKIND kind, uint16_t len, CnetAddr src, char* packet);
  
  // read outgoing message from application to transport layer
  void write_transport(CnetEvent ev, CnetTimerID timer, CnetData data);
  
  // signal transport layer
  void signal_transport(SIGNALKIND sg, SIGNALDATA data);
  \end{lstlisting}
   
\section*{Network layer} 
The purpose of this layer is determining routing and discovering of the
network. When the transport layer has a message to sent, it requests the
network layer for the propagation delay and MTU size along the best possible
path. 

At the stage of neighbor discovery, the network layer advertises itself to other
nodes and based on the answers received and their arrival time builds a routing
table.

Services provided by the layer:
\begin{enumerate}
  \item Determines the routing information for a host
  \item Determines lowest cost path from sender to receiver based on some metric
  (MTU, propagation delay etc.)
  \item Determines propagation delay for routes
  \item Passing message up to the transport layer and down to the datalink layer
\end{enumerate}

    
Top level API:
    
   \begin{lstlisting}
 // initialize network layer
  void init_network();

 // read an incoming packet into network layer
  void write_network(uint8_t kind, CnetAddr dest,uint16_t length, char* data);

 // write an incoming message from datalink to network layer
  void read_network(int link, size_t length, char * datagram);

 // send the datagram to the specified address
  void send_packet(CnetAddr addr, DATAGRAM datagram);

 // send the datagram to the specified link
  void send_packet_to_link(int link, DATAGRAM datagram);

 // broadcast packet to links
  void broadcast_packet(DATAGRAM dtg, int exclude_link);

 //allocated a datagram
  DATAGRAM alloc_datagram(uint8_t prot, int src, int dest, char *data, uint16_t len);

 // clean the memory
  void shutdown_network();

  \end{lstlisting}  
            

\subsection*{Discovery}
Initially nodes are unaware of the addresses of their neighbours. Neither do they know the
attributes of the links they have. In order to learn this information every node advertises
itself to all its neighbours by broadcasting a request and then waiting for replies. Upon receiving 
a request a node sends back a reply containing its address and the propagation delay between the
two nodes. If no response is received from a particular neighbour the node will resend a request to the 
correponding link. 
    
Top level API: 
  \begin{lstlisting}
  //initialize discovery
 void init_discovery();

//process a discovery packet
 void do_discovery(int, DATAGRAM dtg);

// poll our neighbors
 void pollNeighbors();

// discovery timer event handler
 void discovery_timer_handler(CnetEvent, CnetTimerID, CnetData);
\end{lstlisting}
\subsection*{Routing}

The routing algorithm we implement is based upon the AODV (Ad-hoc On-demand
Distance Vector). The nodes are initially unaware of other nodes in the network except 
for their direct neighbours. All nodes maintain a
routing table and a sent datagram history table. The former includes entries for
all destinations already discovered by the node. For every discovered
destination it also contains the entry for the first neighbour to contact. The
latter contains entries for discovery requests that the node has received so
far. Whenever a node needs to send a datagram it first contacts its 
routing table and in case there is an entry for the destination it forwards the 
datagram to the corresponding link. If the destination is not in the routing table
then the node broadcasts the routing request to all its neighbours. A value (hop) is used to
measure the lifetime of the request and is incremented every time the request is
rebroadcast. The node also sets a timer for the request and when it expires the request
is broadcast again with a higher hop. When a datagram arrives at a node it implements 
one of the following three procedures depending on the type of the datagram:
\begin{itemize}
\item \noindent \textbf{Routing request datagram:}
The node first looks it up in its history table and in case there
is already an entry for the datagram it is discarded. If there is no entry then the node 
consults its routing table for the destination address requested and in case it finds it 
the node sends a reply back to the source. This reply is then sent through all nodes it 
has visited in the reverse order until it finally reaches the source. If, on the other hand,
the node does not know a path to the destination, then it increments the hop of the request 
and rebroadcasts the request to all its neighbours. Then it adds a new entry to its history
table which corresponds to the processed request.

\item \noindent \textbf{Routing reply datagram:}
  If the node is itself the source then it simply sends a data datagram it intended to send to the link
the reply has come from (since it now knows that a path to the destination has been discovered and 
this link is the first on the path). If the node is not the source of the original request then
 sends the reply back to the link that it received request from. This way the reply will travel back to
the source.  

\item \noindent \textbf{Data:}
  If a node receives a datagram with a payload then it means that a path to the destination has
already been discovered. Hence the node is either some intermediate node on this path or the 
destination. In the former case it sends the datagram to the next node on the path. If the 
node is the destination then it accepts the datagram, extracts the payload and hands it 
to its transport layer.
\end{itemize}

    Top level API:  
  \begin{lstlisting}
  // initialize the routing
   void init_routing();

  // route a packet
   void route(DATAGRAM);

  // check if a route for specified address exists in routing table
   int route_exists(CnetAddr address);

  //send a route request to find address
   void send_route_request(CnetAddr address, int time_to_live);

  //process a routing packet
   void do_routing(int link,DATAGRAM datagram);

  // learn the routing table
   void learn_route_table(CnetAddr address, int hops, int link,int mtu,CnetTime total_delay);

  // detect a link for outcoming message
   int get_next_link_for_dest(CnetAddr destaddr);

  // detect fragmentation size for the specified address
   int get_mtu(CnetAddr);

  //get propagation delay for specified address
   int get_propagation_delay(CnetAddr);

   void show_table(CnetEvent ev, CnetTimerID timer, CnetData data);

   void shutdown_routing();
  \end{lstlisting}
\newpage

            
\section*{Datalink layer} 
The purpose of this layer is verifying that the datagram arrived is not
corrupted by computing the checksum and passing the datagram further to
the network layer. The layer also maintains a datagram dispatch queue 
and sends the datagrams back to back to the link.
    
    Services provided by the layer:
    \begin{enumerate}
      \item Accepting a datagram from network layer
      \item Passing the datagram up to the network layer 
    \end{enumerate}
    \newpage
    Top level API: 
  \begin{lstlisting}
  // initialize datalink layer
  void init_datalink();
  
  // read an incoming frame into datalink layer
  void read_datalink(CnetEvent event, CnetTimerID timer, CnetData data);
  
  // write an outcoming frame into datalink layer
  void write_datalink(int link, char *datagram, uint32_t length);
  
  // signal transport layer
  void signal_datalink(SIGNALKIND sg, SIGNALDATA data);
  \end{lstlisting} 
 
\section*{Data exchange format}

We define the following basic transfer units: Message, Packet, Datagram.
Every unit encapsulates data, which is the unit from the layer above and header, 
which is the information added by the current layer. 

Kinds of packet and datagram:

    \begin{lstlisting}
    typedef enum {
        ACK = 1, NACK = 2, DATA = 4, LASTSGM = 8,   // transport layer
        DISCOVER = 16, ROUTING = 32, TRANSPORT = 64 // network layer
    } PACKETKIND;
    \end{lstlisting}
    
Kinds of signals used:

    \begin{lstlisting}
    typedef enum {
        CONGESTION = 1, DISCOVERY_FINISHED = 2, MTU_DISCOVERED = 3
    } SIGNALKIND;
    \end{lstlisting}

Transfer unit of application layer:
    \begin{lstlisting}
      char msg[MAX_MESSAGE_SIZE]; // message generated by the application layer
    \end{lstlisting}
  
Transfer unit of transport layer:
       
    \begin{lstlisting}
    typedef struct {
      uint16_t seqno; // datagram sequence number
      uint8_t segid; // segment sequence number
      uint16_t ackseqno; // piggybacked ack sequence number
      uint8_t acksegid; // piggybacked ack segment number
      char msg[MAX_MESSAGE_SIZE]; // encapsulated application layer message
    } PACKET;
    \end{lstlisting}
  
Transfer unit of network layer and datalink layers:   
    
    \begin{lstlisting}
    typedef struct {
      uint8_t src;       // source address
      uint8_t dest;      // destination address
      uint8_t kind;      // packet kind
      uint16_t length;   // length of payload
      uint16_t checksum;  // checksum of entire datagram
      char payload[MAXFRAMESIZE]; // encapsulated packet
    } DATAGRAM;
   \end{lstlisting}
   

\section*{Summary}

\end{document}
